\documentclass{beamer}
\usepackage{minted}

\definecolor{background}{RGB}{39, 40, 34}
\definecolor{string}{RGB}{230, 219, 116}
\definecolor{comment}{RGB}{117, 113, 94}
\definecolor{normal}{RGB}{248, 248, 242}
\definecolor{identifier}{RGB}{166, 226, 46}
\definecolor{keyword}{HTML}{F92672}
\definecolor{numbers}{HTML}{AE81FF}
\definecolor{types}{HTML}{66D9EF}

\usemintedstyle{monokai}
\usetheme{Warsaw}

\addtobeamertemplate{footnote}{\vspace{-6pt}\advance\hsize-0.5cm}{\vspace{6pt}}
\makeatletter
% Alternative A: footnote rule
\renewcommand*{\footnoterule}{\kern -3pt \hrule \@width 2in \kern 8.6pt}
% Alternative B: no footnote rule
% \renewcommand*{\footnoterule}{\kern 6pt}
\makeatother

\setbeamercolor{normal text}{fg=normal,bg=background}
\setbeamercolor{structure}{fg=normal}

\setbeamercolor{alerted text}{fg=red!85!black}

\setbeamercolor{item projected}{use=item,fg=background,bg=item.fg!35}

\setbeamercolor*{palette primary}{use=structure,fg=structure.fg}
\setbeamercolor*{palette secondary}{use=structure,fg=structure.fg!95!black}
\setbeamercolor*{palette tertiary}{use=structure,fg=structure.fg!90!black}
\setbeamercolor*{palette quaternary}{use=structure,fg=structure.fg!95!black,bg=black!80}

\setbeamercolor*{framesubtitle}{fg=normal}

\setbeamercolor*{block title}{parent=structure,bg=background}
\setbeamercolor*{block body}{fg=black,bg=background}
\setbeamercolor*{block title alerted}{parent=alerted text,bg=background}
\setbeamercolor*{block title example}{parent=example text,bg=background}

\title{Welcome to Python}
\subtitle{No Sneks Allowed}
\author{Richard Morrill}
\institute{Fordham University CS Society}
\logo{\includegraphics[width=2cm]{Images/css_logo_color.png}}
\date{Thursday, Novemeber 21st 2019}

\setminted[python]{
  mathescape,
  linenos,
  numbersep=5pt,
  gobble=2,
  % frame=lines,
  framesep=2mm
}
\begin{document}
\begin{frame}
	\titlepage
		    
\end{frame}
\begin{frame}[fragile]
	\frametitle{Hello, Python}
	\begin{minted}{python}
		def main():
		  print("Hello World!")
				        
		if __name__ == "__main__":
		  main()
	\end{minted}
\end{frame}

\section{Setup}
\begin{frame}
  \frametitle{What You Need}
  \begin{itemize}
    \item A Text Editor
    \item A Python Interpreter
  \end{itemize}
  \pause
  Why were your frigging instructions so damn complicated then?
\end{frame}

\begin{frame}
  \frametitle{Why my frigging instructions were so damn complicated}
  \begin{itemize}
    \item Builds Character
    \pause
    \item Docker is Great
    \footnote{Anybody want an event on Docker?}
    \pause
    \item I wanted to remove ALL variables from the process for beginners
    \pause
    \item You could've skipped them anyways
  \end{itemize}
\end{frame}
\begin{frame}
  \frametitle{Setup Steps: Get Resources}
  \begin{itemize}
    \item All resources at: \url{https://github.com/fordham-css/Python_Workshop}
    \item Either use \texttt{git clone} or download it as a zip file
    \item Hint: You can download things from terminal on
          most *NIX systems and in Powershell with \texttt{\$ wget URL}
    \item Navigate to the \texttt{resources} directory
    \item You'll be opening VsCode (After Python / Docker is set up) in this directory,
          by navigating here in Terminal and typing \texttt{\$ code .} or through
          the GUI on Windows (Ask me)\footnote{If you open VsCode in the main folder
          of the repo it will attempt to install a ton of stuff you don't need.}
  \end{itemize}

\end{frame}
\begin{frame}
  \frametitle{Setup Steps: Python}
  \begin{itemize}
    \item Install Text Editor (VsCode Preferred): \url{https://code.visualstudio.com/download}
    \item Install Docker\footnote{Optional}, find correct link in setup Doc
    \pause
    \item If using Docker:
      \begin{itemize}
        \item VsCode Required
        \item Make sure you open it in the right directory\footnote{That's ``resources''}
        \item Install Extension ``Remote - Containers''
        \item Click Orangey Thingy at Bottom Left
        \item Select ``Reopen in Container''
        \item Wait\dots
        \item You're ready
      \end{itemize}
  \end{itemize}
\end{frame}
\begin{frame}
  \frametitle{Setup Steps: Python}

  \begin{itemize}
    \item If not using Docker:
      \begin{itemize}
        \item Install Miniconda: \url{https://docs.conda.io/en/latest/miniconda.html}
        \item Open up:
          \begin{itemize}
            \item Mac: Terminal
            \item Windows: Start Menu $\rightarrow$ Miniconda / Anaconda Prompt
            \footnote{I can't remember exactly what it's called, bug me if it won't work.}
          \end{itemize}
        \pause
        \item Navigate to ``resources'' folder inside wherever you downloaded
              the repo
        \item Open your text editor and wait
      \end{itemize}
  \end{itemize}
\end{frame}
\begin{frame}
  \frametitle{Setup Steps: VsCode}
  \begin{itemize}
    \item If you're using Docker this will already be configured for you
    \item F1
    \item \texttt{> Python: Select Interpreter}
    \item Click the one that that has conda/miniconda somewhere in it
    \footnote{Ask for help if you don't see it}
    \item F1
    \item \texttt{> Python: Select Linter}
    \item Click ``pylint''
    \item It will prompt you to install PyLint, let it\footnote{If it asks
          whether you want to use conda or pip say conda}
  \end{itemize}
\end{frame}
\begin{frame}
  \frametitle{You did it!}
  You now should all have a working Python setup.  We'll continue once
  everybody has gotten it working.
\end{frame}
\section{Basic Syntax}
\begin{frame}
  \frametitle{Python Has all the Basics}
  \begin{itemize}
    \item It's got functions
    \pause
    \item It's got variables
    \pause
    \item It's got operators
    \pause
    \item It's got classes and OOP\footnote{Not covered today}
  \end{itemize}
\end{frame}
\begin{frame}
  \frametitle{Variables}
  \begin{itemize}
    \item For those that haven't programmed before:
      \begin{itemize}
        \item Not ``unknowns'' like in math
        \item A place to store pieces of information
      \end{itemize}
    \item Open \texttt{basics/variables.py}
  \end{itemize}
  \pause

\end{frame}
\begin{frame}[fragile]
  \frametitle{Variables}
  \begin{minted}{python}
    # This file demonstrates variables in Python
    # You don't need to use a 'var' keyword
    # or declare a type
    a = 1
    b = "A string"
    c = {
      "This is": "A dictionary",
      "It's pretty much": "the same thing as a JS object"
    }
    # This is a list
    d = [a, b, c]
    # This is fine, you can change type on the fly
    a = b
    \end{minted}
\end{frame}
\begin{frame}[fragile]
  \frametitle{Printing}
  \begin{itemize}
    \item It's nice to see the results of our program.
    \item We do this (in basic programs) by \underline{printing} to the
          terminal.
    \item Anybody know why it's called that?
    \pause
  \end{itemize}
  \begin{minted}{python}
  print("A string")
  print(1)
  print([1, 4, 5, 6])
  \end{minted}
  \begin{itemize}
    \item You can print nearly anything, with varying degrees of success
    \item Try printing some things, see what works well and what doesn't
  \end{itemize}
\end{frame}
\begin{frame}
  \frametitle{Control Flow}
  \begin{itemize}
    \item In its simplest form, a Python script is just a list of instructions
          the interpreter does sequentially.
    \pause
    \item You control the order in which it executes statements using:\footnote{Who wants to tell me what each of these is?}
    \begin{itemize}
      \item Conditionals
      \item Loops
      \item Functions
    \end{itemize}
  \end{itemize}

\end{frame}
\end{document}