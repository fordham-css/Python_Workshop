\documentclass[aspectratio=169]{beamer}
\usepackage{minted}

\definecolor{background}{RGB}{39, 40, 34}
\definecolor{string}{RGB}{230, 219, 116}
\definecolor{comment}{RGB}{117, 113, 94}
\definecolor{normal}{RGB}{248, 248, 242}
\definecolor{identifier}{RGB}{166, 226, 46}
\definecolor{keyword}{HTML}{F92672}
\definecolor{numbers}{HTML}{AE81FF}
\definecolor{types}{HTML}{66D9EF}

\usemintedstyle{monokai}
\usetheme{Warsaw}
\useoutertheme{infolines}
\addtobeamertemplate{footnote}{\vspace{-6pt}\advance\hsize-0.5cm}{\vspace{6pt}}
\makeatletter
% Alternative A: footnote rule
\renewcommand*{\footnoterule}{\kern -3pt \hrule \@width 2in \kern 8.6pt}
% Alternative B: no footnote rule
% \renewcommand*{\footnoterule}{\kern 6pt}
\makeatother

\setbeamercolor{normal text}{fg=normal,bg=background}
\setbeamercolor{structure}{fg=normal}

\setbeamercolor{alerted text}{fg=red!85!black}

\setbeamercolor{item projected}{use=item,fg=background,bg=item.fg!35}

\setbeamercolor*{palette primary}{use=structure,fg=structure.fg}
\setbeamercolor*{palette secondary}{use=structure,fg=structure.fg!95!black}
\setbeamercolor*{palette tertiary}{use=structure,fg=structure.fg!90!black}
\setbeamercolor*{palette quaternary}{use=structure,fg=structure.fg!95!black,bg=black!80}

\setbeamercolor*{framesubtitle}{fg=normal}

\setbeamercolor*{block title}{parent=structure,bg=background}
\setbeamercolor*{block body}{fg=black,bg=background}
\setbeamercolor*{block title alerted}{parent=alerted text,bg=background}
\setbeamercolor*{block title example}{parent=example text,bg=background}

\title{Welcome to Python}
\subtitle{No Sneks Allowed}
\author{Richard Morrill}
\institute{Fordham University CS Society}
\logo{\includegraphics[width=2cm]{Images/css_logo_color.png}}
\date{Thursday, Novemeber 21st 2019}

\setminted[python]{
  mathescape,
  linenos,
  numbersep=5pt,
  autogobble,
  % frame=lines,
  framesep=2mm
}
\begin{document}
\begin{frame}
	\titlepage
		    
\end{frame}
\begin{frame}[fragile]
	\frametitle{Hello, Python}
	\begin{minted}{python}
		def main():
		  print("Hello World!")
				        
		if __name__ == "__main__":
		  main()
	\end{minted}
\end{frame}
\begin{frame}
  \frametitle{Why Python?}
  \begin{itemize}
    \item Python is an interpreted programming language / scripting language (i.e. an \underline{interpreter} runs your code)
    \pause
    \begin{itemize}
      \item No need to write ``boilerplate'' like in C++
      \item Much harder to crash your computer
      \item Much harder to leave critical security vulnerabilities\footnote{Well, of a very specific type\dots}
    \end{itemize}
    \pause
    \item Python is EVERYWHERE
    \pause
    \begin{itemize}
      \item Machine Learning
      \item Data Science
      \item Some random utility some guy wrote for one very specific problem
    \end{itemize}
    \pause
    \item Python is Great for Personal Projects
    \pause
    \begin{itemize}
      \item Easy to start fast
      \item No need to worry about advanced CS stuff
    \end{itemize}
  \end{itemize}
\end{frame}
\begin{frame}
  \frametitle{Why not Python?}
  \pause
  It's SLOW AF
  
  \pause
  It hides a lot of the ``fun'' from you
\end{frame}
\section{Setup}
\begin{frame}
  \frametitle{What You Need}
  \begin{itemize}
    \item A Text Editor
    \item A Python Interpreter
  \end{itemize}
  \pause
  Why were your frigging instructions so damn complicated then?
\end{frame}

\begin{frame}
  \frametitle{Why my frigging instructions were so damn complicated}
  \begin{itemize}
    \item Builds Character
    \pause
    \item Docker is Great
    \footnote{Anybody want an event on Docker?}
    \pause
    \item I wanted to remove ALL variables from the process for beginners
    \pause
    \item You could've skipped them anyways
  \end{itemize}
\end{frame}
\begin{frame}
  \frametitle{Setup Steps: Get Resources}
  \begin{itemize}
    \item All resources at: \url{https://github.com/fordham-css/Python_Workshop}
    \item Either use \texttt{git clone} or download it as a zip file
    \item Hint: You can download things from terminal on
          most *NIX systems and in Powershell with \texttt{\$ wget URL}
    \item Navigate to the \texttt{resources} directory
    \item You'll be opening VsCode (After Python / Docker is set up) in this directory,
          by navigating here in Terminal and typing \texttt{\$ code .} or through
          the GUI on Windows (Ask me)\footnote{If you open VsCode in the main folder
          of the repo it will attempt to install a ton of stuff you don't need.}
  \end{itemize}

\end{frame}
\begin{frame}
  \frametitle{Setup Steps: Python}
  \begin{itemize}
    \item Install Text Editor (VsCode Preferred): \url{https://code.visualstudio.com/download}
    \item Install Docker\footnote{Optional}, find correct link in setup Doc
    \pause
    \item If using Docker:
      \begin{itemize}
        \item VsCode Required
        \item Make sure you open it in the right directory\footnote{That's ``resources''}
        \item Install Extension ``Remote - Containers''
        \item Click Orangey Thingy at Bottom Left
        \item Select ``Reopen in Container''
        \item Wait\dots
        \item You're ready
      \end{itemize}
  \end{itemize}
\end{frame}
\begin{frame}
  \frametitle{Setup Steps: Python}

  \begin{itemize}
    \item If not using Docker:
      \begin{itemize}
        \item Install Miniconda: \url{https://docs.conda.io/en/latest/miniconda.html}
        % \item Open up:
        %   \begin{itemize}
        %     \item Mac: Terminal
        %     \item Windows: Start Menu $\rightarrow$ Miniconda / Anaconda Prompt
        %     \footnote{I can't remember exactly what it's called, bug me if it won't work.}
        %   \end{itemize}
        % \pause
        \item Navigate to ``resources'' folder inside wherever you downloaded
              the repo
        \item Open your text editor and wait
      \end{itemize}
  \end{itemize}
\end{frame}
\begin{frame}
  \frametitle{Setup Steps: VsCode}
  \begin{itemize}
    \item If you're using Docker this will already be configured for you
    \item F1
    \item \texttt{> Python: Select Interpreter}
    \item Click the one that that has conda/miniconda somewhere\footnote{Ask for help if you don't see it}
    \item F1
    \item \texttt{> Python: Select Linter}
    \item Click ``pylint''
    \item It will prompt you to install PyLint, let it\footnote{If it asks
          whether you want to use conda or pip say conda}
  \end{itemize}
\end{frame}
\begin{frame}
  \frametitle{You did it!}
  You now should all have a working Python setup.  We'll continue once
  everybody has gotten it working.

  \vspace{24pt}
  To ensure it actually is working:
  \begin{itemize}
    \item Open a terminal (VsCode menu bar $\rightarrow$ Terminal $\rightarrow$ New Terminal)
    \item Type \texttt{python --version}
    \item You should see ``Python 3.7.5''\footnote{Don't sweat it if your number is
    slightly lower}
  \end{itemize}
\end{frame}
\section{Basic Syntax}
\begin{frame}
  \frametitle{Python Has all the Basics}
  \begin{itemize}
    \item It's got functions
    \pause
    \item It's got variables
    \pause
    \item It's got operators
    \pause
    \item It's got classes and OOP\footnote{Not covered today}
  \end{itemize}
\end{frame}
\begin{frame}
  \frametitle{Indented Syntax}
  \begin{itemize}
    \item If you're used to C++ this will trip you up
    \item Python relies on indent to tell which code is in
          which block.
    \item You may use either tabs or spaces, but you must be consistent
    \item Keep top-level code right against the left margin
    \item This will all make more sense once you see examples
  \end{itemize}
\end{frame}
\begin{frame}
  \frametitle{Variables}
  \begin{itemize}
    \item For those that haven't programmed before:
      \begin{itemize}
        \item Not ``unknowns'' like in math
        \item A place to store pieces of information
      \end{itemize}
    \item Open \texttt{basics/variables.py}
  \end{itemize}
  \pause

\end{frame}
\begin{frame}[fragile]
  \frametitle{Variables}
  \begin{minted}{python}
    # This file demonstrates variables in Python
    # You don't need to use a 'var' keyword
    # or declare a type
    a = 1
    b = "A string"
    c = {
      "This is": "A dictionary",
      "It's pretty much": "the same thing as a JS object"
    }
    # This is a list
    d = [a, b, c]
    # This is fine, you can change type on the fly
    a = b
    \end{minted}
\end{frame}
\begin{frame}[fragile]
  \frametitle{Printing}
  \begin{itemize}
    \item It's nice to see the results of our program.
    \item We do this (in basic programs) by \underline{printing} to the
          terminal.
    \item Anybody know why it's called that?
    \pause
  \end{itemize}
  \begin{minted}{python}
  print("A string")
  print(1)
  print([1, 4, 5, 6])
  \end{minted}
  \begin{itemize}
    \item You can print nearly anything, with varying degrees of success
    \item Try printing some things, see what works well and what doesn't
  \end{itemize}
\end{frame}
\begin{frame}
  \frametitle{Control Flow}
  \begin{itemize}
    \item In its simplest form, a Python script is just a list of instructions
          the interpreter does sequentially.
    \pause
    \item You control the order in which it executes statements using:\footnote{Who wants to tell me what each of these is?}
    \begin{itemize}
      \item If Statements
      \item Loops
      \item Functions
    \end{itemize}
  \end{itemize}
\end{frame}
\begin{frame}[fragile]
  \frametitle{Conditionals}
  \begin{itemize}
    \item There's another variable type: \underline{boolean} (True / False)
    \item You can define them literally:
  \end{itemize}
  \begin{minted}{python}
  condition = True
  if condition:
    print("If statements are pretty easy to deal with")
  else:
    print("Just don't forget the colons and indent")
  \end{minted}
  \pause
  \begin{itemize}
    \item Or use the result of a comparison\footnote{I'll go over all the available comparisons later}
  \end{itemize}
  \begin{minted}{python}
  if "something" == "not something":
    print("This will never execute")
  else:
    print("This will always execute")
  \end{minted}
\end{frame}
\begin{frame}[fragile]
  \frametitle{Functions}
  \begin{itemize}
    \item For those that haven't programmed before:
    \begin{itemize}
      \item A set of statements that can be run over and over again
      \item \underline{Called} by name
      \item Optionally takes input and returns output
    \end{itemize}
  \end{itemize}
  \pause
  \begin{minted}{python}
  def example_function(var_a, var_b):
    '''This function takes 2 arguments'''

    # It also returns 2 values
    return var_b, var_a
  \end{minted}
\end{frame}
\begin{frame}[fragile]
  \frametitle{Functions}
  \begin{minted}{python}
  def no_arguments():
    # Do stuff...
    return 5
  
  def no_return(var_1, var_2):
    # Do stuff
    var_1 += 3
    print(var_2 + var_1)
    # Notice there's no return statement
  
  # If you want, you can still return early from a function
  def print_conditional(var_1):
    if (var_1 == 5):
      return
    print(var_1)
  \end{minted}
\end{frame}
\begin{frame}
  \frametitle{Loops}
  \begin{itemize}
    \item Do the same thing over and over
    \pause
    \item While Loops:
    \begin{itemize}
      \item Run until condition is no longer true
      \item Won't even run once if condition isn't true
    \end{itemize}
    \pause
    \item For Loops:
      \begin{itemize}
        \item Run a fixed number of times\footnote{Although that number can be modified while they run, and they can exit early}
        \item In Python almost always over a list of something
      \end{itemize}
  \end{itemize}
\end{frame}
\begin{frame}[fragile]
  \frametitle{While Loop Example}
  \begin{minted}{python}
  b = 5
  while(b > 0):
    b = b - 1
  \end{minted}
  \pause
  \begin{itemize}
    \item Anybody see the potential issue here?
    \pause
    \item While loops always need a good exit condition
  \end{itemize}
\end{frame}
\begin{frame}[fragile]
  \frametitle{For Loop Example}

  \begin{minted}{python}
  a = [5, 7, 8]
  for num in a:
    print(num)
  \end{minted}
  \pause
  If you want to exit early
  \begin{minted}{python}
    a = [5, 7, 8]
    for num in a:
      if num == 7:
        break
      print(num)
  \end{minted}
\end{frame}
\begin{frame}[fragile]
  \frametitle{For Loop Example}

  If you want to run a fixed number of times without a list:
  \begin{minted}{python}
    for i in range(9):
    print(i)
  \end{minted}

\end{frame}
\section{Math}
\begin{frame}
  \frametitle{Numbers}
  \begin{itemize}
    \item Most languages have a bunch of number types
    \pause
    \item Python only has 3 (that you can see):
    \begin{itemize}
      \item int
      \item float
      \item complex
    \end{itemize}
  \end{itemize}
\end{frame}
\begin{frame}[fragile]
  \frametitle{Numbers}
  \begin{minted}{python}
  # An integer
  a = 4
  print(a)
  # A floating point number
  b = 4.3
  print(b)
  # Round down a float
  print(int(4.3))
  # A string
  c = "1000"
  print(c * 5)
  \end{minted}
  \pause
  $\rightarrow$Somebody run that last line
  \pause
  \begin{minted}{python}
  # So, convert a string to a number
  print(int(c) * 5)
  \end{minted}
\end{frame}
\begin{frame}[fragile]
  \frametitle{Arithmatic Operators}
  \begin{minted}{python}
  # All of these work exactly like you'd expect
  print(2 + 3)
  print(10 / 2)
  print(3 * 7)
  print(3 - 19)
  # If you're used to C++, this might work differently
  # than you'd expect
  print(11 / 2)
  # If you want integer division, use //
  print(11 // 2)
  # Python also has a cool exponentiation operator
  print("3^3 = {}".format(3**3))
  \end{minted}
\end{frame}
\begin{frame}
  \frametitle{Logical Operators}
  \begin{itemize}
    \item Used to compare values:
      \begin{itemize}
        \item Equality: \mintinline{python}{ 3 == 5}
        \item Greater / Less Than: \mintinline{python}{ 2 > 4}
        \item Greater Than or Equal: \mintinline{python}{ 5 >= 8}
      \end{itemize}
      \pause
    \item Combining logical statements
    \begin{itemize}
      \item And: \mintinline{python}{2 < 8 and 4 == 4}
      \item Or: \mintinline{python}{"a" == "b" or 5 == 5}
    \end{itemize} 
  \end{itemize}
\end{frame}
\section{Strings and Lists}
\begin{frame}
  \frametitle{What is a string?}
  \begin{itemize}
    \item If you haven't programmed before:
      \begin{itemize}
        \item A sequence of \underline{characters}
        \item Used to store text
      \end{itemize}
    \pause
    \item Closely related to lists
  \end{itemize}
\end{frame}
\begin{frame}[fragile]
  \frametitle{String Operations}
  \begin{minted}{python}
a = "This is a string"
b = "this is another string"
# We can concatenate them together
c = a + ", and " + b
# We can split them apart
for w in a.split():
    print(w)
# We can change case
print(a.upper())
print(a.lower())
# we can test if a string contains another
if "is" in a:
    print("Yay!")
else:
    print("Boo!")
  \end{minted}
\end{frame}
\begin{frame}
  \frametitle{What is a list?}
  \begin{itemize}
    \item An ordered collection of values
    \item Holds any type
  \end{itemize}
\end{frame}
\begin{frame}[fragile]
  \frametitle{List Operations}
  \begin{minted}{python}
    a = [66.25, 333, 333, 1, 1234.5]
    print a.count(333), a.count(66.25), a.count('x')
    # 1st Argument index of element before which to insert
    a.insert(2, -1)
    a.append(333)
    # Find index of first item with that value
    a.index(333)
    # Remove item with that value
    a.remove(333)
    a.reverse()
    a.sort()
    a.pop()
  \end{minted}
\end{frame}
\section{Conclusions}
\begin{frame}
  \frametitle{You Have the Tools}
  \begin{itemize}
    \item All I've given you today is the tools, it's up to you to figure
          out how to actually use them
    \pause
    \item Project Ideas:
      \begin{itemize}
        \item Simple Calculator
        \item Guessing Game
      \end{itemize}
      \pause
    \item Possible Future Events
      \begin{itemize}
        \item Scikit (Data Science)
        \item Game Making
        \item Tensorflow (Neural Networks)
      \end{itemize}
  \end{itemize}

  

\end{frame}
\end{document}