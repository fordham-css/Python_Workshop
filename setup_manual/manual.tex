\documentclass[12pt]{article}
\usepackage{hyperref}
\author{Richard Morrill\\
\small{Fordham University CS Society}}
\parskip=12pt
\begin{document}
    \section{Introduction}
    Thanks for taking an interest in my Python workshop!  I'm thrilled
    you want to learn Python.  Whether you're an experienced programmer
    in another language, or a complete beginner, I think you'll find Python
    to be a worthwhile addition to your toolbox.  This manual is going to
    take you through the basic ``housekeeping'' steps to make sure you have
    an identical environment to the one I used to write and test the code
    in this tutorial.  Note that the setup steps may seem overly complicated
    for simply writing basic Python\dots and they are!  The reason for this
    is that by setting up a robust environment, we'll be able to avoid putting
    out too many fires at the beginning of the workshop, and also I think
    that getting exposure to more capable tools from the start will make
    your learning curve with Python quite a bit easier.  I anticipate that
    these steps will take 10-20 minutes.

    \section{Code Editor}
    The editor I use for all languages is Microsoft's VsCode.  If you use
    a different editor for coding, it probably works fine for Python, and
    you can keep using it, you just might not get robust code linting,
    completion, and debugging features. Also, I'll be less equipped to help
    you fix any issues you might have.  So, use anything other than VsCode
    at your own risk.

    Install VsCode here: \url{https://code.visualstudio.com/}

    \section{Python Environment}
    There are two ways to set up an environment so that you can run the code
    for this event.  The first (and simplest) way is through Docker. The main
    advantage of doing it this way is that everything runs in a container, a
    self-contained environment totally configured by scripts I already wrote
    for you.  This will ensure that 1) nothing you do will break anything
    you normally use, and 2) if you screw something up you can just rebuild
    the container and it'll be back in working order.  If you are not
    comfortable using a command line I highly, highly recommend just using
    Docker as it will basically just configure everything for you.  Also,
    Docker is a great tool to have as a programmer, you'll probably want to
    learn how to use it at some point.

    \subsection{Docker}
    How you install Docker depends on your operating system.  Please see
    the appropriate subsection.  No matter which version you install, at
    no point should you actually need to create a Docker account.  Technically
    you're supposed to make one before you get to the download link, but I've
    conveniently skipped that step for you and put the links directly in this
    document.

    When you get to the workshop, I'll provide you with a repository that will
    include the Dockerfile and config files needed to allow you to instantly
    begin writing Python inside a container. (i.e. You're done once you've
    installed Docker.)

    \subsubsection{Windows}
    The version of Docker you need depends on whether you have a
    \emph{pro} or \emph{home} version of Windows.  If you aren't sure,
    (on Windows 10 only), right click on the Start Menu icon, click System,
    then read the ``Windows specifications'' section.

    If you have Windows 10 Pro: \url{https://download.docker.com/win/stable/Docker%20for%20Windows%20Installer.exe}
    you will be installing \emph{Docker Desktop}.

    If you have Windows 10 Home: \url{https://docs.docker.com/toolbox/toolbox_install_windows/}.
    You will be installing \emph{Docker Toolbox}.  This program has the same functionality
    as Docker Desktop, but the installation is a little  bit more complicated.  Please
    follow the instructions at the link and let me know if you have any questions.
    \subsubsection{Linux}
    If you use Linux on your daily driver I shouldn't need to tell you how
    to install Docker\dots NEXT!

    \subsubsection{Mac}
    Ugh\dots if you've been coming to my events you know that Apple can be
    the WORST when it comes to setting up a development environment that
    doesn't revolve around XCode.  Luckily the Docker installer takes care
    of all the weird bits for you this time. \url{https://download.docker.com/mac/stable/Docker.dmg}
    \section{Directly Installing Python}

    If Docker won't work for you, or you have something against containers,
    you can simply install Python on your system.  If you know what you're
    doing, just install Python and Pylint, and I can help you make sure
    VsCode recognizes your Python install.  If you haven't done something
    like this before, I recommend installing Miniconda: \url{https://docs.conda.io/en/latest/miniconda.html}.
    Simply follow the instructions for your operating system, and then I'll take you
    through the rest of the setup steps at the Workshop.
\end{document}